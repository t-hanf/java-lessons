\input{../templates/course_definitions}
% This Document contains the information about this course.

% Authors of the slides
\author{Tobias Hanf, Manik Khurana}

% Name of the Course
\institute{Java-Course}

% Fancy Logo
\titlegraphic{\hfill\includegraphics[height=1.25cm]{../templates/fsr_logo_cropped}}


\title{Java}
\subtitle{Scopes \& GUI}
\date{\today}

\begin{document}

\begin{frame}
\titlepage
\end{frame}

\begin{frame}{Overview}
\tableofcontents
\end{frame}

\section{Recursion in Java}
\subsection{What is Recursion?}
\begin{frame}[fragile]{Recursion}
	The process in which a function calls itself directly or indirectly is called recursion and the corresponding function is called as recursive function. Using recursive algorithm, certain problems can be solved quite easily. Examples - Towers of Hanoi (TOH), Inorder/Preorder/Postorder Tree Traversals, DFS of Graph, etc.
\end{frame}
\begin{frame}[fragile]{Recursion Contd.}
	Approach 1:
	\begin{lstlisting}
		approach(1) - Simply adding one by one
		f(n) = 1 + 2 + 3 + .... + n 
	\end{lstlisting}
    Approach 2:
    \begin{lstlisting}
        approach(2) - Mathematical expression 
    	f(n) = 1             n=1
    	f(n) = n + f(n-1)    n>1
    \end{lstlisting}
\end{frame}
\subsection{Examples}
\begin{frame}[fragile]{Example}
	In the recursive program, the solution to the base case is provided and the solution of the bigger problem is expressed in terms of smaller problems.
	\begin{lstlisting} 
	int fact(int n)
	{
	if (n < = 1) // base case
	return 1;
	else    
	return n*fact(n-1);    
	}	
	\end{lstlisting}
	In the above example, the base case is defined and larger value of number can be solved by converting to smaller one till base case is reached.
\end{frame}

\begin{frame}[fragile]{Recursion Example}
	\begin{lstlisting} 
	// A Java program to demonstrate working of recursion
	class GFG {
		static void printFun(int test)
		{
			if (test < 1)
			return;
			else {
				System.out.printf("%d ", test);
				printFun(test - 1); // statement 2
				System.out.printf("%d ", test);
				return;
			}}
		// Driver Code
		public static void main(String[] args){
			int test = 3;
			printFun(test);
		}
	}
	\end{lstlisting}
\end{frame}
\subsection{Recursion flow in program}
\begin{frame}[fragile]{Flow of program}
\begin{center}
	\includegraphics[width=\textwidth]{09-scopes_gui/0001.jpg}
\end{center}
\end{frame}

\begin{frame}[fragile]{Why Stack Overflow error occurs in recursion?}
	Happens if the base case is not reached or not defined.
	\begin{lstlisting} 
		int fact(int n)
		{
			// wrong base case (it may cause stack overflow).
			if (n == 100) 
			return 1;
			else
			return n*fact(n-1);
		}
	\end{lstlisting}
	If fact(10) is called, it will call fact(9), fact(8), fact(7) and so on but the number will never reach 100. So, the base case is not reached. If the memory is exhausted by these functions on the stack, it will cause a stack overflow error. 
\end{frame}

\begin{frame}[fragile]{Programming exercise}
	Refer to fib.java code
\end{frame}

\section{Scopes}
\subsection{Classes}

\begin{frame}[fragile]{Visiabilities}
	\begin{lstlisting}
	class MyGreatClass {
	
		//Attributes are public by default
		Car myCar; 
	
		//Public are available in every part of our code.
		public Cat myCat;
		
		//Private Attributes can only be acced via a method
		private House myHouse;
	}
		
	\end{lstlisting}
\end{frame}

\subsection{Controll Structures}
\begin{frame}[fragile]{For}
	\begin{lstlisting}
	for(int i = 0; i <= 100; i++) {
		int b = 3;
		System.out.println(i);
		System.out.println(b);
	}
	\end{lstlisting}
	
	b will be redefined in every round of the loop and is only available in the for loop.
	
	The scope is created at the begining and destroyed at the end of each round.
\end{frame}
  
\subsection{Mulit definitions}
\begin{frame}[fragile]{Examples}
	\begin{lstlisting}
	public class myClass {
		private int a;
	
		public myClass(int a) {
			this.a = a;
		}
	}
	\end{lstlisting}

	Use nearest definition.

	In one scope every variable name can be defined only one time.
\end{frame}
    
\section{GUI}
\subsection{Window}
\begin{frame}[fragile]{Window}
  Windows are created by creating a \texttt{JFrame} object.
	\begin{lstlisting}
		// Create a new window
		JFrame window = new JFrame();

		// Set its title and size
		window.setTitle("Guestbook");
		window.setSize(500, 500);
    
    // Show the window
    window.setVisible(true);
	\end{lstlisting}
\end{frame}

\begin{frame}[fragile]{Window}
  You can add panels and other elements to the window.
	\begin{lstlisting}
		// Add a blue background panel
		JPanel backgroundPanel = new JPanel();
		backgroundPanel.setBackground(Color.BLUE);
		window.add(backgroundPanel);
	\end{lstlisting}

  More advanced panels are available (JSplitPane, JScrollPane, JTabbedPane etc..)
\end{frame}

\subsection{Menus}
\begin{frame}[fragile]{Menus}
  \texttt{Menus} are used to display multiple possible actions to the user.
	\begin{lstlisting}
		// Menu Bar holds all menus
    MenuBar bar = new MenuBar();
    
    // Create new menu which holds menu items and submenus
		Menu menu = new Menu("File");

    // Create new menu item 
		MenuItem item = new MenuItem("Create new User");
    
    // Combine everything
		menu.add(item);
		bar.add(menu);
		window.setMenuBar(bar);
	\end{lstlisting}
\end{frame}

\subsection{Actions}
\begin{frame}[fragile]{Actions}
  In order to respond to button presses or other changes in the UI, you need to add listeners
	\begin{lstlisting}
    // Create menu item
		MenuItem item = new MenuItem("Create new User");
  
    // Add listener
		item.addActionListener(new ActionListener() {
			@Override
			public void actionPerformed(ActionEvent e) {
				userManager.addUser();
			}
		});
  \end{lstlisting}
\end{frame}
    
\begin{frame}[fragile]{Actions}
  Another example for a list
	\begin{lstlisting}
    // Create list with single selection option.
		this.userList = new JList();
		this.userList.setSelectionMode(ListSelectionModel.SINGLE_SELECTION);
      
    // Add listener
		this.userList.addListSelectionListener(new ListSelectionListener() {
			@Override
			public void valueChanged(ListSelectionEvent e) {
				int selectedIndex = this.userList.getSelectedIndex();
			}
		});
    \end{lstlisting}
\end{frame}
\end{document}