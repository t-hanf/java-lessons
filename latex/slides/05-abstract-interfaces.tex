\input{../templates/course_definitions}
% This Document contains the information about this course.

% Authors of the slides
\author{Tobias Hanf, Manik Khurana}

% Name of the Course
\institute{Java-Course}

% Fancy Logo
\titlegraphic{\hfill\includegraphics[height=1.25cm]{../templates/fsr_logo_cropped}}


\usepackage{hyperref}

\title{Java}
\subtitle{Abstract and Interfaces}
\date{\today}

\begin{document}
	
	\begin{frame}
		\titlepage
	\end{frame}
	\begin{frame}{Overview}
		\setbeamertemplate{section in toc}[sections numbered]
		\tableofcontents
	\end{frame}

\section{Abstract}
\subsection{}
\begin{frame}[fragile]{Abstract Class}
	The keyword \textbf{abstract} denotes an abstract class.
	\vfill
	\begin{lstlisting}
		public abstract class AbstractExample {
			
		}	
	\end{lstlisting}
	\vfill
	\begin{itemize}
		\item You can not create objects from an abstract class.\\
		\item Abstract classes can extend other abstract classes and can implement interfaces \footnote[1]{Interfaces will be discussed later}.\\
		\item Abstract classes can be extended by normal and abstract classes.
	\end{itemize}
\end{frame}

\begin{frame}[fragile]{Methods}
	An abstract class may has concrete methods and may has abstract methods. It can also has fields (attributes).
	\begin{lstlisting}
		public abstract class AbstractExample {
			
			public void printHello() {
				System.out.println("Hello");	    
			}
			
			public abstract String getName();
		}	
	\end{lstlisting}
	An abstract method forces the class to be abstract as well. \\
	%\emph{Methods in an interface are also abstract but not denoted as such.}
\end{frame}

\begin{frame}[fragile]{Subclasses}
	The subclass has to implement abstract methods or has to be abstract as well.
	All concrete methods will be regular inherited.
	\begin{lstlisting}[escapechar=!]
		public class Example extends AbstractExample {
			
			@Override
			public String getName() {
				return "Example";	    
			}
		}	
	\end{lstlisting}
\end{frame}

\section{Polymorphism}
\begin{frame}{Types}
	Definition: providing a single interface to entities of different types \footnote{\url{https://www.stroustrup.com/glossary.html\#Gpolymorphism}}
	
	Two different types of polymorphism in Java:
	\begin{itemize}
		\item function/operator overloading
		\item subtyping (polymorphism)
	\end{itemize}
\end{frame}

\begin{frame}[fragile]{Function overloading}
	Having the same function name but different parameters (types or number)
	\begin{lstlisting}[escapechar=!]
		public class Calc {
			
			double Add(double a, double b){
				System.out.println("Adding doubles");
				return a + b;
			}
		
			int Add(int a, int b){
				System.out.println("Adding integers");
				return a + b;
			}
		}	
	\end{lstlisting}
\end{frame}

\begin{frame}[fragile]{Function overloading}
	Calling the function with different types.
	\begin{lstlisting}[escapechar=!]
		public class Main {
			
			public static main(String[] args){
				Calc calc = new Calc();
				
				calc.Add(3.5, 4.6);
				// prints: Adding doubles
				
				calc.Add(3, 4);
				// prints: Adding integers
			}
		}	
	\end{lstlisting}
	The compiler will check the types of the parameters an select the right function based on the signature.
\end{frame}

\begin{frame}[fragile]{Polymorphism}
	\begin{lstlisting}[escapechar=!]
		public class Person {	
			...
			
			public void sendEmail(String message){
				System.out.println("To: " this.email 
				+ " \n" + message);
			}
		}
		
		public class Student extends Person {
			@Override
			public void sendEmail(String message){
				System.out.println("To: " this.email +
				 " \n" + "Dear student :" + message);
			}	
		}
	\end{lstlisting}
	
\end{frame}

\begin{frame}[fragile]{Polymorphism}
	Using the interface of the super class
	\begin{lstlisting}[escapechar=!]
		public class Main {
			public static main(String[] args){
				Person p1 = new Person("Bob");
				Person p2 = new Student("Alice");
				
				p1.sendEmail("Hello");
				// prints: 	To: bob@school.org
				//			Hello
				
				p2.sendEmail("Hello");
				// prints: 	To: alice@school.org
				//			Dear student: Hello	
			}	
		}
	\end{lstlisting}
	A object of type \texttt{Student} can be assigned to a variable of type \texttt{Person} because \texttt{Student} is a subclass of \texttt{Person}.
	\end{frame}


\section{Static}

\subsection{Overview}
%TODO improve this overview - add an example - think about the headline
\begin{frame}{Static Keyword}
	An object is an instance of a class with its attributes and methods.
	The object is the actor and the class just a blueprint.
	\vfill
	Static class members are not linked to a certain instance of the class.
	Therefore the class can also be an actor.
	\vfill
	Static class members are:
	\begin{itemize}
		\item static attributes, often called class variables
		\item static methods, often called class methods
	\end{itemize}
\end{frame}

\subsection{Examples}
\begin{frame}[fragile]{Class Variables}
	In the setter \texttt{count} is addressed via \texttt{Example.count}.
	Using \texttt{this.count} is misleading, because \texttt{count} is a class variable.	
	\begin{lstlisting}
		public class Example {
			
			public static count;
			
			public setCount(int count) {
				Example.count = count;
			}
		}
	\end{lstlisting}
\end{frame}

\begin{frame}[fragile]{Class Variables - Test}
	The test prints the class variable \texttt{Example.count} which is altered
	by the different instances of the class \emph{Example}.	
	\begin{lstlisting}[basicstyle=\ttfamily\scriptsize]
		public class ExampleTest {
			
			public static void main (String[] args) {
				Example e1 = new Example();
				Example e2 = new Example();
				
				e1.setCount(4);
				System.out.println(Example.count); // prints: 4
				e2.setCount(8);
				System.out.println(Example.count); // prints: 8
			}
		}
	\end{lstlisting}
\end{frame}

\begin{frame}[fragile]{Class Methods}
	Static methods can be called without an object.
	They can modify class variables but not attributes (object variables).
	\begin{lstlisting}
		public class Example {
			
			public static count;
			
			public static void setCount(int count) {
				Example.count = count;
			}
		}
	\end{lstlisting}
	\begin{lstlisting}
		public static void main (String[] args) {
			
			Example.setCount(4);
		}
	\end{lstlisting}
\end{frame}

\subsection{Implications}
\begin{frame}[fragile]{Static is an One-Way}
	Methods from objects can:
	\begin{itemize}
		\item access attributes (object variables)
		\item access class variables
		\item call methods
		\item call static methods
	\end{itemize}
	Class methods can:
	\begin{itemize}
		\item access class variables
		\item call static methods
	\end{itemize}
\end{frame}

\section{Interfaces}
\subsection{Overview}
\begin{frame}{}
	An \textbf{interface} is a well defined set of constants and methods a class have to \textbf{implement}.
	\vfill
	You can access objects through their interfaces. So you can work with different kinds of objects easily.
	\vfill
	For Example: A post office offers to ship letters, postcards and packages. With an interface
	\emph{Trackable} you can collect the positions unified. 
	It is not important how a letter calculates its position.
	%TODO rethink the following sentence
	It is important that the letter communicate its position through the methods from the interface.
	%vfill
	%conclusion: Seperation from definition and implementation. 
	%The definition is ?denoted by? the interface.
\end{frame}
\subsection{Example}
\begin{frame}[fragile]{Interface Trackable}
	An interface contains method signatures. A signature is the definition of a method without the implementation.
	\begin{lstlisting}
		public interface Trackable {
			
			public int getStatus(int identifier);
			
			public Position getPosition(int identifier);
		}
	\end{lstlisting}
	Note: The name of an interface often ends with the suffix \emph{-able}.
\end{frame}
\begin{frame}[fragile]{Letter implements Trackable}
	\begin{lstlisting}
		public class Letter implements Trackable {
			
			public Position position;
			private int identifier;
			
			public int getStatus(int identifier) {
				return this.identifier;
			}
			
			public Position getPosition(int identifier) {
				return this.position;
			}
		}
	\end{lstlisting}
	The classes \emph{Postcard} and \emph{Package} also implement the interface \emph{Trackable}.
\end{frame}

\begin{frame}[fragile]{Access through an Interface}
	\begin{lstlisting}
		public static void main(String[] args) {
			
			Trackable letter_1 = new Letter();
			Trackable letter_2 = new Letter();
			Trackable postcard_1 = new Postcard();
			Trackable package_1 = new Package();
			
			letter_1.getPosition(2345);
			postcard_1.getStatus(1234);
		}
	\end{lstlisting}
\end{frame}

\subsection{Multiple Interfaces}
\begin{frame}[fragile]{Two Interfaces}
	A class can implement multiple interfaces.
	\vfill
	\begin{lstlisting}[basicstyle=\ttfamily\scriptsize]
		public interface Buyable {
			
			// constant
			public float tax = 1.19f;
			
			public float getPrice();
		}
	\end{lstlisting}
	\begin{lstlisting}[basicstyle=\ttfamily\scriptsize]
		public interface Trackable {
			
			public int getStatus(int identifier);
			
			public Position getPosition(int identifier);
		}
	\end{lstlisting}
\end{frame}

\begin{frame}[fragile]{Postcard implements Buyable and Trackable}
	\begin{lstlisting}[basicstyle=\ttfamily\scriptsize]
		public class Postcard implements Buyable, Trackable {
			
			public Position position;
			private int identifier;
			private float priceWithoutVAT;
			
			public float getPrice() {
				return priceWithoutVAT * tax;
			}
			
			public int getStatus(int identifier) {
				return this.identifier;
			}
			
			public Position getPosition(int identifier) {
				return this.position;
			}
		}
	\end{lstlisting}
\end{frame}

\begin{frame}[fragile]{Access multiple Interfaces}
	\begin{lstlisting}
		public static void main(String[] args) {
			
			Trackable postcard_T = new Postcard();
			Postcard postcard_P = new Postcard();
			Buyable postcard_B = new Postcard();
			
			postcard_T.getStatus(1234);
			postcard_B.getPrice();
			postcard_P.getStatus(1234);
			postcard_P.getPrice();
		}
	\end{lstlisting}
	\texttt{postcard\_P} can access both interfaces.\\
	\texttt{postcard\_T} can access Trackable.\\
	\texttt{postcard\_B} can access Buyable.
\end{frame}
\end{document}