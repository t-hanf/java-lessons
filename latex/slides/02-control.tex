\input{../templates/course_definitions}
% This Document contains the information about this course.

% Authors of the slides
\author{Tobias Hanf, Manik Khurana}

% Name of the Course
\institute{Java-Course}

% Fancy Logo
\titlegraphic{\hfill\includegraphics[height=1.25cm]{../templates/fsr_logo_cropped}}


\title{Java}
\subtitle{Control Statements and OOP}
\date{\today}

\lstdefinestyle{base}{
	%language=C,
	%emptylines=1,
	%breaklines=true,
	%basicstyle=\ttfamily\color{black},
	moredelim=**[is][\color{red}]{@}{@},
}

\begin{document}

\begin{frame}
	\titlepage
\end{frame}

\begin{frame}{Overview}
	\setbeamertemplate{section in toc}[sections numbered]
	\tableofcontents
\end{frame}


\section{Recalling last session}

\begin{frame}[fragile]{Blocks}
	\begin{lstlisting}[style=base]
		public class Hello @{@
			// prints a "Hello World!" on your console
			public static void main(String[] args) {
				System.out.println("Hello World!");
			}
			@}@
	\end{lstlisting}
	Everything between \{ and \} is a \emph{block}. \\
	Blocks may be nested.
\end{frame}

\begin{frame}[fragile]{Naming of Variables}
	\begin{itemize}
		\item The names of variables can begin with any letter or underscore. \\
		Usually the name starts with small letter.
		\item Compound names should use CamelCase.
		\item Use meaningful names.
	\end{itemize}
	\begin{lstlisting}
		public class Calc {
			public static void main(String[] args) {
				int a = 0; // not very meaningful
				float myFloat = 5.3f; // also not meaningfull
				int count = 7; // quite a good name
				
				int rotationCount = 7; // there you go
			}
		}
	\end{lstlisting}
\end{frame}

\begin{frame}{Primitive data types}
	Java supports some primitive data types:
	\begin{itemize}
		\item[boolean] a truth value (either \textbf{true} or \textbf{false})
		\item[int] a 32 bit integer
		\item[long] a 64 bit integer
		\item[float] a 32 bit floating point number
		\item[double] a 64 bit floating point number
		\item[char] an unicode character
		\item[void] the empty type (needed in later topics)
	\end{itemize}
\end{frame}

\subsection{Calculating}

\begin{frame}[fragile, allowframebreaks]{Calculating with \emph{int}}
	\begin{lstlisting}
		public class Calc {
			public static void main(String[] args) {
				int a;
				a = 7;
				System.out.println(a);
				a = 8;
				System.out.println(a);
				a = a + 2;
				System.out.println(a);
			}
		}
	\end{lstlisting}
	
	\framebreak
	\begin{lstlisting}
		public class Calc {
			public static void main(String[] args) {
				int a; // declare variable a
				a = 7; // assign 7 to variable a
				System.out.println(a); // prints: 7
				a = 8;
				System.out.println(a); // prints: 8
				a = a + 2;
				System.out.println(a); // prints: 10
			}
		}
	\end{lstlisting}
	After the first assignment the variable is initialized.
	\framebreak
	\begin{lstlisting}
		public class Calc {
			public static void main(String[] args) {
				int a = -9;
				int b;
				b = a;
				System.out.println(a);
				System.out.println(b); 
				a++;
				System.out.println(a); 
			}
		}
	\end{lstlisting}
	\framebreak
	\begin{lstlisting}
		public class Calc {
			public static void main(String[] args) {
				int a = -9; // declaration and assignment of a
				int b; // declaration of b
				b = a; // assignment of b
				System.out.println(a); // prints: -9
				System.out.println(b); // prints: -9
				a++; // increments a
				System.out.println(a); // prints: -8
			}
		}
	\end{lstlisting}
	% \framebreak
	% 	\begin{lstlisting}
		% 	public class Calc {
			% 	    public static void main(String[] args) {
				% 	        int b; // declaration of b
				% 	        System.out.println(b);
				% 	    }
			% 	}
		% 	\end{lstlisting}
	% 	Uninitialized variables will cause an Exception. \\
	% 	An Exception is a kind of error we will discuss later.\\
	% 	\vspace{1em}
	% 	\emph{Always assign your variables!}
	% \framebreak
	
	\framebreak
	Some basic mathematical operations:
	\begin{tabular}{ll}
		Addition & \texttt{a + b;} \\
		Subtraction & \texttt{a - b;} \\
		Multiplication &\texttt{a * b;} \\
		Division & \texttt{a / b;} \\
		Modulo & \texttt{a \% b;} \\
		Increment & \texttt{a++;} \\
		Decrement & \texttt{a--;} \\
	\end{tabular}
\end{frame}

\begin{frame}[fragile, allowframebreaks]{Calculating with \emph{float}}
	\begin{lstlisting}
		public class Calc {
			public static void main(String[] args) {
				float a = 9;
				float b = 7.5f;
				System.out.println(a); // prints: 9.0
				System.out.println(b); // prints: 7.5
				System.out.println(a + b); // prints: 16.5
			}
		}
	\end{lstlisting}
	
	\framebreak
	\begin{lstlisting}
		public class Calc {
			public static void main(String[] args) {
				float a = 0.1f;
				float b = 0.2f;
				
				System.out.println(((a + b) == 0.3));
			}
		}
	\end{lstlisting}
	
	
	\framebreak
	\begin{lstlisting}
		public class Calc {
			public static void main(String[] args) {
				float a = 0.1f;
				float b = 0.2f;
				
				System.out.println(((a + b) == 0.3)); // false
				System.out.println((a + b));
			}
		}
	\end{lstlisting}
	Float has a limited precision. \\
	\emph{This might lead to unexpected results!}
\end{frame}

\begin{frame}[fragile]{Mixing \emph{int} and \emph{float}}
	\begin{lstlisting}
		public class Calc {
			public static void main(String[] args) {
				float a = 9.3f;
				int b = 3;
				System.out.println(a + b); // prints: 12.3
				float c = a + b;
				System.out.println(c); // prints: 12.3
			}
		}
	\end{lstlisting}
	Java converts from \textbf{int} to \textbf{float} by default, if necessary. \\
	But not vice versa.
\end{frame}

\subsection{Text with Strings}

\begin{frame}[fragile]{Strings}
	A String is not a primitive data type but an object. \\
	We discuss objects in detail in the next section.
	\begin{lstlisting}
		public class Calc {
			public static void main(String[] args) {
				String hello = "Hello World!";
				System.out.println(hello); // print: Hello World!
			}
		}
	\end{lstlisting}
\end{frame}

\begin{frame}[fragile]{Concatenation}
	\begin{lstlisting}
		public class Calc {
			public static void main(String[] args) {
				String hello = "Hello";
				String world = " World!";
				String sentence = hello + world;
				System.out.println(sentence);
				System.out.println(hello + " World!");
			}
		}
	\end{lstlisting}
	You can concatenate Strings using the +. Both printed lines look the same.
\end{frame}

\begin{frame}[fragile]{Strings and Numbers}
	\begin{lstlisting}
		public class Calc {
			public static void main(String[] args) {
				int factorA = 3;
				int factorB = 7;
				int product = factorA * factorB;
				String answer =
				factorA + " * " + factorB + " = " + product;
				System.out.println(answer); // prints: 3 * 7 = 21
			}
		}
	\end{lstlisting}
	Upon concatenation, primitive types will be replaced by their current value as \emph{String}.
\end{frame}

\begin{frame}{Conclusion}
	Datatypes
	\begin{itemize}
		\item int, long
		\item float, double
		\item String
	\end{itemize}
	Hello World example
\end{frame}

\section{Input}
\begin{frame}[fragile, allowframebreaks]{Taking input from \emph{the User} }
	
	To take input from the user we use the Scanner class. 
	For this we need to import the java.util.Scanner package. 
	
	\begin{lstlisting}
		import java.util.Scanner; 
		public class Input
		{
			public static void main(String[] args)
			{
				//do something
			}
		}
	\end{lstlisting}
\framebreak

	To use the Scanner class we create a new Scanner object.\\
	//not telling you what an object means at this stage
	
	\begin{lstlisting}
		import java.util.Scanner; 
		public class Input
		{
			public static void main(String[] args)
			{
			Scanner sc = new Scanner(System.in);
			System.out.print("Please input a number: ");
			int a = sc.nextInt();
			System.out.println("Input number = "+a);
				
			}
		}
	\end{lstlisting}

\framebreak

To use the Scanner class we create a new Scanner object.\\
//For educational purposes only

\begin{lstlisting}
	import java.util.Scanner; 
	public class Input
	{
		public static void main(String[] args)
		{
		Scanner sc = new Scanner(System.in);
		System.out.print("Please input a number: ");
		int a = sc.nextInt();
		System.out.println("Input number = "+a);
		
		System.out.print("Please input a decimal number: ");
		Double b = sc.nextDouble();
		System.out.println("Input number = "+b);
		}
	}
\end{lstlisting}

\framebreak

\begin{lstlisting}
	import java.util.Scanner; 
	public class Input
	{
		public static void main(String[] args)
		{
		Scanner sc = new Scanner(System.in);
		System.out.print("Please input a number: ");
		int a = sc.nextInt();
		System.out.println("Input number = "+a);
		System.out.print("Please input a decimal number: ");
		Double b = sc.nextDouble();
		System.out.println("Input number = "+b);
			System.out.print("Please input a String: ");
			String c = sc.nextLine();
			System.out.println("Input String = "+c);	
		}
	}
\end{lstlisting}

\end{frame}



\section{Control Statements}
\begin{frame}{Control Statements}
	
	\begin{itemize}
		\item if, else, else if
		\item for
		\item while
	\end{itemize}
		
\end{frame}



\subsection{Ite}
\begin{frame}[fragile]{{\huge I}f {\huge T}hen {\huge E}lse}
\begin{lstlisting}
if(condition) {
	// do something if condition is true
} else if(another condition){
	// do if "else if" condition is true 
} else {
	// otherwise do this
}
\end{lstlisting}
\end{frame}

\begin{frame}[fragile]{{\huge I}f {\huge T}hen {\huge E}lse example}
\begin{lstlisting}
public class IteExample {

	public static void main(String[] args) {
		int myNumber = 5;
		
		if(myNumber == 3) {
			System.out.println("Strange number");
		} else if(myNumber == 2) {
			System.out.println("Unreachable code");
		} else {
			System.out.println("Will be printed");
		}
	}
    
}
\end{lstlisting}
\end{frame}

\begin{frame}{Conditions?}
How to compare things:
\begin{itemize}
    \item \texttt{==} Equal
    \item \texttt{!=} Not Equal
    \item \texttt{>} Greater Than
    \item \texttt{>=} Greater or Equal than
\end{itemize}
\textit{Note}: You can concatenate multiple conditions\\ with \texttt{\&\&} (AND) or \texttt{||} (OR)
\end{frame}

\subsection{for}
\begin{frame}[fragile]{for}
\begin{lstlisting}
for(initial value, condition, change) {
	// do code while condition is true
}
\end{lstlisting}
\end{frame}

\begin{frame}[fragile]{for example}
\begin{lstlisting}
public class ForExample {

	public static void main(String[] args) {
		for(int i = 0; i <= 10; i++) {
			System.out.print("na ");
		}
		System.out.println("BATMAN!");
	}
    
}
\end{lstlisting}
\end{frame}

\subsection{while}
\begin{frame}[fragile]{while}
\begin{lstlisting}
while(condition) {
	// do code while condition is true
}
\end{lstlisting}
\end{frame}

\begin{frame}[fragile]{while example}
\begin{lstlisting}
public class WhileExample {

	public static void main(String[] args) {
		int a = 0;
		while(a <= 10) {
			System.out.println(a);
            a++; // Otherwise you would get an endless loop
		}
	}
    
}
\end{lstlisting}
\end{frame}

\section{Methods}
\begin{frame}{What is a method?}
	What is a method?
	\begin{itemize}
		\item<1-> piece of code which can be reused
		\item<2-> can take input from the outside
		\item<3-> can return data to the outside
	\end{itemize}
	\onslide<4->{Other names:
	\begin{itemize}
		\item function
		\item procedure
		\item subroutine
	\end{itemize}
}
\end{frame}

\begin{frame}{Why use methods?}
	Why use methods?
	\begin{itemize}
		\item<1-> programmers are lazy $\rightarrow$ do more with less code
		\item<2-> better structure and less changes
		\item<3-> reduces errors
		\item<4-> important for OOP
	\end{itemize}
	
\end{frame}

\begin{frame}[fragile, allowframebreaks] {Introduction to \emph{methods}}
	The most basic method
	\begin{lstlisting}
	static void helloMethod(){
		System.out.println("Hello, method!");	
	}
	\end{lstlisting}
\framebreak
	Calling the method in main
	\begin{lstlisting}
	class Hello{
		public static main(String[] args){
			helloMethod();
		}
		static void helloMethod(){
			System.out.println("Hello, method!");	
		}
	}
	\end{lstlisting}

\framebreak
	Giving data into a function (Parameters)
	\begin{lstlisting}
		static void printHello(String input){
			System.out.println("Hello, " + input + "!");	
		}
	\end{lstlisting}

\framebreak
	Returning data from the function (Return)
	\begin{lstlisting}
		static String getHello(String input){
			String hello = "Hello, " + input + "!";	
			return hello;
			// return "Hello, " + input + "!";
		}
	\end{lstlisting}

\framebreak
DEMO
	

\end{frame}

\begin{frame}{Exercise}
	Print all even numbers between 1 and 100/n (take user input)
	\begin{itemize}
		\item<2-> with while and if
		\item<3-> with for (without if)
		\item<4-> with a function to check evenness
		
	\end{itemize}
\end{frame}


\begin{frame}
	\begin{center}
		\includegraphics[angle=-90, width=\textwidth]{0001}
	\end{center}	

\end{frame}



\end{document}