\input{../templates/course_definitions}
% This Document contains the information about this course.

% Authors of the slides
\author{Tobias Hanf, Manik Khurana}

% Name of the Course
\institute{Java-Course}

% Fancy Logo
\titlegraphic{\hfill\includegraphics[height=1.25cm]{../templates/fsr_logo_cropped}}


\title{Java}
\subtitle{Controll Statements and OOP}
\date{\today}

\begin{document}
	
	\begin{frame}
		\titlepage
	\end{frame}
	\begin{frame}{Overview}
		\setbeamertemplate{section in toc}[sections numbered]
		\tableofcontents
	\end{frame}
	
	\section{Recalling last session}
	\begin{frame}{Conclusion}
		Control Structures
		\begin{itemize}
			\item taking input
			\item If-Then-Else
			\item for and while loop
			\item Conditions
		\end{itemize}
		Na Na Na Na Na Na Na Na Batman
	\end{frame}	
	
	\section{OOP in Java}
	
	\begin{frame}{}
		\begin{center}
			{\huge Object Oriented Programming}
		\end{center}
	\end{frame}

	\subsection{Main concept}
	\begin{frame}[fragile]{What is OOP?}
		\begin{itemize}
			\item OOP - Object Oriented Programming
			\item want to model the real world
			\item take things and create digital copy
			\item two main concepts - Objects and Classes
		\end{itemize}	
	\end{frame}
	\begin{frame}[fragile]{What is a class?}
		A blueprint for a series of objects with common attributes/methods \\
		Example Car: \\
		\begin{itemize}
			\item<2-> attributes \\ \begin{itemize}
					\item wheels
					\item windows
					\item color
					\item engine
					\item ...
					
				\end{itemize}
			\item<3-> methods \\ 
						\begin{itemize}
							\item accelerate
							\item break
							\item toggle turn signal
							\item ...
						\end{itemize}
		\end{itemize}
		
	\end{frame}

	\begin{frame}[fragile]{What is an object?}
		An object is an instantiation of a class
		\begin{itemize}
			\item a class does not really exist
			\item only objects exist
			\item taking a class and filling it with data
			\item can be created and destroyed
		\end{itemize} 
	\end{frame}
	
	\subsection{General information}
	
	\begin{frame}[fragile]{Class Student}
		\begin{lstlisting}
			public class Student {
				
				// Attributes
				private String name; 
				private int matriculationNumber; 
				
				
				// Methods
				public void setName(String name) {
					this.name = name;
				}
				
				public int getMatriculationNumber() {
					return matriculationNumber;
				}
				
			}
		\end{lstlisting}
		
		% What is visible here:
		% Attributes store the state of the object
		% Methods implement the behaviour of the object
		
	\end{frame}

	\begin{frame}{private, public, protected?}
		\begin{itemize}
			\item<1-> there is an outside and an inside
			\item<2-> not everything should be visible from the outside
			\item<3-> have to tell the compiler what to show and what not
			\item<4-> keywords \\
				\begin{itemize}
					\item private - only visible from inside the class
					\item public - visible from everywhere
					\item protected - visible to child classes
				\end{itemize}
			
			\item<5-> attributes should always be private (or protected)
			\item<6-> methods which are for internal use should be private
			\item<7-> every other methods can/should be public
		\end{itemize}
	\end{frame}

	\begin{frame}[fragile]{Class Student}
		\begin{lstlisting}
			public class Student {
				
				// Attributes
				private String name; 
				private int matriculationNumber; 
				
				
				// Methods
				public void setName(String name) {
					this.name = name;
				}
				
				public int getMatriculationNumber() {
					return matriculationNumber;
				}
				
			}
		\end{lstlisting}
		
		% What is visible here:
		% Attributes store the state of the object
		% Methods implement the behaviour of the object
		
	\end{frame}
	
	\begin{frame}[fragile]{Creation}
		We learned how to declare and assign a primitive datatype.
		
		\begin{lstlisting}
			int a; // declare a
			a = 273; // assign 273 to a
		\end{lstlisting} 
		
		The creation of an object works similar.
		
		\begin{lstlisting}
			Student example = new Student(); 
			// create an instance of Student
		\end{lstlisting}
		The \textbf{object} derived from a \textbf{class} is also called \textbf{instance}.
		The variable is called the \textbf{reference}.
	\end{frame}

	\begin{frame}[fragile]{Class Student}
		\begin{lstlisting}
			public class Student {
				
				// Attributes
				private String name; 
				private int matriculationNumber; 
				
				
				// Methods
				public void setName(String name) {
					this.name = name;
				}
				
				public int getMatriculationNumber() {
					return matriculationNumber;
				}
				
			}
		\end{lstlisting}
		
		% What is visible here:
		% Attributes store the state of the object
		% Methods implement the behaviour of the object
		
	\end{frame}

	\begin{frame}{What is a method?}
		What is a method?
		\begin{itemize}
			\item<1-> piece of code which can be reused
			\item<2-> can take input from the outside
			\item<3-> can return data to the outside
		\end{itemize}
		\onslide<4->{Other names:
			\begin{itemize}
				\item function
				\item procedure
				\item subroutine
			\end{itemize}
		}
	\end{frame}
	
	\begin{frame}{Why use methods?}
		Why use methods?
		\begin{itemize}
			\item<1-> programmers are lazy $\rightarrow$ do more with less code
			\item<2-> better structure and less changes
			\item<3-> reduces errors
			\item<4-> important for OOP
		\end{itemize}
		
	\end{frame}
	
	\subsection{Methods}
	\begin{frame}[fragile]{Calling a Method}
		\begin{lstlisting}
			public class Student {
				
				private String name;
				
				public String getName() {
					return name;
				}
				
				public void setName(String newName) {
					name = newName;
				}
				
			}
		\end{lstlisting}
		The class \emph{Student} has two methods: \emph{void setName()} and \emph{String getName()}.
	\end{frame}
	
	\begin{frame}[fragile]{Calling a Method}
		\begin{lstlisting}
			public class Main {
				
				public static void main(String[] args) {
					Student example = new Student(); // creation
					example.setName("Jane"); // method call
					String name = example.getName(); 
					System.out.println(name); // Prints "Jane"
				}
				
			}
		\end{lstlisting}
		You can call a method of an object after its creation with \textbf{reference.methodName();}.
	\end{frame}
	
	\begin{frame}[fragile]{Calling a Method}
		\begin{lstlisting}
			public class Student {
				
				private String name;
				
				public void setName(String newName) {
					name = newName;
					printName();   // Call own method
					this.printName(); // Or this way
				}
				
				public void printName() {
					System.out.println(name);
				}
				
			}
		\end{lstlisting}
		You can call a method of the own object by simply writing \textbf{methodName();} or \textbf{this.methodName();}
	\end{frame}
	
	\begin{frame}[fragile]{Methods with Arguments}
		
		\begin{lstlisting}
			public class Calc {
				
				public void add(int summand1, int summand2) {
					System.out.println(summand1 + summand2);
				}
				
				public static void main(String[] args) {
					int summandA = 1;
					int summandB = 2;
					Calc calculator = new Calc();
					System.out.print("1 + 2 = ");
					calculator.add(summandA, summandB); 
					// prints: 3
				}
				
			}
		\end{lstlisting}
	\end{frame}
	
	\subsection{Return Value}
	\begin{frame}[fragile]{Methods with Return Value}
		A method without a return value is indicated by \textbf{void}:
		\begin{lstlisting}
			public void add(int summand1, int summand2) {
				System.out.println(summand1 + summand2);
			}
		\end{lstlisting}
		A method with an \textbf{int} as return value:
		\begin{lstlisting}
			public int add(int summand1, int summand2) {
				return summand1 + summand2;
			}
		\end{lstlisting}
		% TODO explain return statement
	\end{frame}
	
	\begin{frame}[fragile]{Calling Methods with a return value}
		\begin{lstlisting}
			public class Calc {
				
				public int add(int summand1, int summand2) {
					return summand1 + summand2;
				}
				
				public static void main(String[] args) {
					Calc calculator = new Calc();
					int sum = calculator.add(3, 8);
					System.out.print("3 + 8 = " + sum); 
					// prints: 3 + 8 = 11
				}
				
			}
		\end{lstlisting}
	\end{frame}
	
	\subsection{Constructor}
	
	\begin{frame}[fragile]{Constructors}
		\begin{lstlisting}
			public class Calc {
				
				private int summand1;
				private int summand2;
				
				public Calc() {
					summand1 = 0;
					summand2 = 0;
				}
				
			}
		\end{lstlisting}
		A constructor gets called upon creation of the object
	\end{frame}
	
	\begin{frame}[fragile]{Constructors with Arguments}
		\begin{lstlisting}
			public class Calc {
				
				private int summand1;
				private int summand2;
				
				public Calc(int x, int y) {
					summand1 = x;
					summand2 = y;
				}
				
			}
		\end{lstlisting}
		\begin{lstlisting}
			[...]
			Calc myCalc = new Calc(7, 9);
		\end{lstlisting}
		
		A constructor can have arguments as well!
	\end{frame}
	
	\section{Conclusion}
	\subsection{An Example}
	
	\begin{frame}{An Example}
		You want to program an enrollment system, for a programming course. \\
		\vspace{1em}
		Your classes are:\\
		\begin{description}
			\item[student] who wants to attend the course
			\item[lesson] which is a part of the course
			\item[tutor] the guy with the bandshirt
			\item[room] where your lessons take place
			\item[\dots]
		\end{description}
		% 	The more you think about it, the more complex this program becomes.
		% 	Focus on the relevant things.
		%	Think about how the objects can be in relation, this will be discussed later
		%	Show prepared classes in Java
	\end{frame}
	
	
	\begin{frame}[fragile]{Class Student}
		\begin{lstlisting}
			public static void main(String[] args) {
				Student peter = new Student();
				peter.changeName("Peter");
			}
		\end{lstlisting}
	\end{frame}
	
\end{document}