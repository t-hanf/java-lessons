\input{../templates/course_definitions}
% This Document contains the information about this course.

% Authors of the slides
\author{Tobias Hanf, Manik Khurana}

% Name of the Course
\institute{Java-Course}

% Fancy Logo
\titlegraphic{\hfill\includegraphics[height=1.25cm]{../templates/fsr_logo_cropped}}


\title{Java}
\subtitle{Kontrollstatements and OOP}
\date{\today}

\lstdefinestyle{base}{
	%language=C,
	%emptylines=1,
	%breaklines=true,
	%basicstyle=\ttfamily\color{black},
	moredelim=**[is][\color{red}]{@}{@},
}

\begin{document}

\begin{frame}
	\titlepage
\end{frame}

\begin{frame}{Übersicht}
	\setbeamertemplate{section in toc}[sections numbered]
	\tableofcontents
\end{frame}


\section{Wiederholung }

\begin{frame}[fragile]{Blocks}
	\begin{lstlisting}[style=base]
		public class Hello @{@
			// gibt "Hello, world!" in der Konsole aus
			public static void main(String[] args) {
				System.out.println("Hello World!");
			}
			@}@
	\end{lstlisting}
	Alles zwischen \{ und \} ist Inhalt eines \emph{Blockes}. \\
	Blöcke können geschachtelt sein.
\end{frame}

\begin{frame}[fragile]{Benennung von Variablen}
	\begin{itemize}
		\item Ein Variablenname kann mit einem Buchstaben oder einen \_ Underscore beginnen. \\
		Normalerweise beginnen sie mit einem kleinen Buchstaben.
		\item Für zusammengesetzte Namen verwendet man die CamleCase-Notation.
		\item Es sollten immer aussagekraeftige Namen verwendet werden.
	\end{itemize}
	\begin{lstlisting}
		public class Calc {
			public static void main(String[] args) {
				int a = 0; // nicht sehr aussagekraeftig
				float myFloat = 5.3f; // auch nicht sehr aussagekraeftig
				int count = 7; // ein besserer Name
				
				int rotationCount = 7; // so sollte es sein
			}
		}
	\end{lstlisting}
\end{frame}

\begin{frame}{Primitive Datentypen}
	Java unterstützt einige primitive Datentypen:
	\begin{itemize}
		\item[boolean] ein Wahrheitswert (entweder \textbf{true} oder \textbf{false})
		\item[int] a 32 bit integer (ganze Zahl)
		\item[long] a 64 bit integer
		\item[float] a 32 bit floating point number (Gleitkommazahl)
		\item[double] a 64 bit floating point number
		\item[char] an unicode character
		\item[void] leerer Typ (für Später)
	\end{itemize}
\end{frame}

\subsection{Rechnen}

\begin{frame}[fragile, allowframebreaks]{Rechnen mit \emph{int}}
	\begin{lstlisting}
		public class Calc {
			public static void main(String[] args) {
				int a;
				a = 7;
				System.out.println(a);
				a = 8;
				System.out.println(a);
				a = a + 2;
				System.out.println(a);
			}
		}
	\end{lstlisting}
	
	\framebreak
	\begin{lstlisting}
		public class Calc {
			public static void main(String[] args) {
				int a; // deklariert die Variable a
				a = 7; // weis a den Wert 7 zu
				System.out.println(a); // prints: 7
				a = 8;
				System.out.println(a); // prints: 8
				a = a + 2;
				System.out.println(a); // prints: 10
			}
		}
	\end{lstlisting}
	Nach der ersten Zuweisung ist eine Variable vollständig initialisiert.
	\framebreak
	\begin{lstlisting}
		public class Calc {
			public static void main(String[] args) {
				int a = -9;
				int b;
				b = a;
				System.out.println(a);
				System.out.println(b); 
				a++;
				System.out.println(a); 
			}
		}
	\end{lstlisting}
	\framebreak
	\begin{lstlisting}
		public class Calc {
			public static void main(String[] args) {
				int a = -9; // Deklaration mit Zuweisung
				int b; // declaration of b
				b = a; // assignment of b
				System.out.println(a); // prints: -9
				System.out.println(b); // prints: -9
				a++; // increments a
				System.out.println(a); // prints: -8
			}
		}
	\end{lstlisting}
	% \framebreak
	% 	\begin{lstlisting}
		% 	public class Calc {
			% 	    public static void main(String[] args) {
				% 	        int b; // declaration of b
				% 	        System.out.println(b);
				% 	    }
			% 	}
		% 	\end{lstlisting}
	% 	Uninitialized variables will cause an Exception. \\
	% 	An Exception is a kind of error we will discuss later.\\
	% 	\vspace{1em}
	% 	\emph{Always assign your variables!}
	% \framebreak
	
	\framebreak
	Einige grundlegende Rechenoperationen:
	\begin{tabular}{ll}
		Addition & \texttt{a + b;} \\
		Subtraction & \texttt{a - b;} \\
		Multiplication &\texttt{a * b;} \\
		Division & \texttt{a / b;} \\
		Modulo & \texttt{a \% b;} \\
		Increment & \texttt{a++;} \\
		Decrement & \texttt{a--;} \\
	\end{tabular}
\end{frame}

\begin{frame}[fragile, allowframebreaks]{Rechnen mit \emph{float}}
	\begin{lstlisting}
		public class Calc {
			public static void main(String[] args) {
				float a = 9;
				float b = 7.5f;
				System.out.println(a); // prints: 9.0
				System.out.println(b); // prints: 7.5
				System.out.println(a + b); // prints: 16.5
			}
		}
	\end{lstlisting}
	
	\framebreak
	\begin{lstlisting}
		public class Calc {
			public static void main(String[] args) {
				float a = 0.1f;
				float b = 0.2f;
				
				System.out.println(((a + b) == 0.3));
			}
		}
	\end{lstlisting}
	
	
	\framebreak
	\begin{lstlisting}
		public class Calc {
			public static void main(String[] args) {
				float a = 0.1f;
				float b = 0.2f;
				
				System.out.println(((a + b) == 0.3)); // false
				System.out.println((a + b));
			}
		}
	\end{lstlisting}
	Float besitzt nur eine begrenzte Genauigkeit. \\
	\emph{Dies kann zu unvorhersehbaren Ergebnissen führen!}
\end{frame}

\begin{frame}[fragile]{Rechnnen mit \emph{int} und \emph{float}}
	\begin{lstlisting}
		public class Calc {
			public static void main(String[] args) {
				float a = 9.3f;
				int b = 3;
				System.out.println(a + b); // prints: 12.3
				float c = a + b;
				System.out.println(c); // prints: 12.3
			}
		}
	\end{lstlisting}
	Java konvertiert automatisch von \textbf{int} zu \textbf{float}, wenn dies nötig ist. \\
	Aber nicht andersherum.
\end{frame}

\subsection{Text mit Strings}

\begin{frame}[fragile]{Strings}
	Ein String ist ein primitiver Datentyp. Es ist ein Objekt \\
	Objekte werden in der nächsten Stunde besprochen.
	\begin{lstlisting}
		public class Calc {
			public static void main(String[] args) {
				String hello = "Hello World!";
				System.out.println(hello); // print: Hello World!
			}
		}
	\end{lstlisting}
\end{frame}

\begin{frame}[fragile]{Concatenation}
	\begin{lstlisting}
		public class Calc {
			public static void main(String[] args) {
				String hello = "Hello";
				String world = " World!";
				String sentence = hello + world;
				System.out.println(sentence);
				System.out.println(hello + " World!");
			}
		}
	\end{lstlisting}
	Man Strings mit den + Operator aneinander hängen. Dadurch sehen beide Zeilen gleich aus.
\end{frame}

\begin{frame}[fragile]{Strings und Zahlen}
	\begin{lstlisting}
		public class Calc {
			public static void main(String[] args) {
				int factorA = 3;
				int factorB = 7;
				int product = factorA * factorB;
				String answer =
				factorA + " * " + factorB + " = " + product;
				System.out.println(answer); // prints: 3 * 7 = 21
			}
		}
	\end{lstlisting}
	Wenn ein primitiver Datentyp mit einen String verbunden wird, wird dieser in eine \emph{String} umgewandelt.
\end{frame}

\begin{frame}{Zusammenfassung}
	Datentypen
	\begin{itemize}
		\item int, long
		\item float, double
		\item String
	\end{itemize}
	Hello World Beispiel
\end{frame}

\section{Input}
\begin{frame}[fragile, allowframebreaks]{Eingabe vom \emph{Benutzer} }
	
	Um eine Eingabe von einen Nutzer zu lesen benötigen wir ein Objekt vom Typ Scanner. Dafür müssen wir das Paket java.util.Scanner importieren.
	
	\begin{lstlisting}
		import java.util.Scanner; 
		public class Input
		{
			public static void main(String[] args)
			{
				//do something
			}
		}
	\end{lstlisting}
\framebreak

	Um den Scanner zu benutzen, müssen wir zuerst ein Objekt von der Klasse Scanner erzeugen (??? wird in der nächsten Stunde besprochen).
	
	\begin{lstlisting}
		import java.util.Scanner; 
		public class Input
		{
			public static void main(String[] args)
			{
			Scanner sc = new Scanner(System.in);
			System.out.print("Please input a number: ");
			int a = sc.nextInt();
			System.out.println("Input number = "+a);
				
			}
		}
	\end{lstlisting}

\framebreak

Um den Scanner zu benutzen, müssen wir zuerst ein Objekt von der Klasse Scanner erzeugen. (Beispiel-Code, bitte nicht in realen Programmen verwenden)

\begin{lstlisting}
	import java.util.Scanner; 
	public class Input
	{
		public static void main(String[] args)
		{
		Scanner sc = new Scanner(System.in);
		System.out.print("Please input a number: ");
		int a = sc.nextInt();
		System.out.println("Input number = "+a);
		System.out.print("Please input a decimal number: ");
		Double b = sc.nextDouble();
		System.out.println("Input number = "+b);
		}
	}
\end{lstlisting}

\framebreak

\begin{lstlisting}
	import java.util.Scanner; 
	public class Input
	{
		public static void main(String[] args)
		{
		Scanner sc = new Scanner(System.in);
		System.out.print("Please input a number: ");
		int a = sc.nextInt();
		System.out.println("Input number = "+a);
		System.out.print("Please input a decimal number: ");
		Double b = sc.nextDouble();
		System.out.println("Input number = "+b);
			System.out.print("Please input a String: ");
			String c = sc.nextLine();
			System.out.println("Input String = "+c);	
		}
	}
\end{lstlisting}

\end{frame}



\section{Control Statements}
\begin{frame}{Control Statements}
	
	\begin{itemize}
		\item if, else, else if
		\item for
		\item while
	\end{itemize}
		
\end{frame}



\subsection{Ite}
\begin{frame}[fragile]{{\huge I}f {\huge T}hen {\huge E}lse}
\begin{lstlisting}
if(condition) {
	// do something if condition is true
} else if(another condition){
	// do if "else if" condition is true 
} else {
	// otherwise do this
}
\end{lstlisting}
\end{frame}

\begin{frame}[fragile]{{\huge I}f {\huge T}hen {\huge E}lse example}
\begin{lstlisting}
public class IteExample {

	public static void main(String[] args) {
		int myNumber = 5;
		
		if(myNumber == 3) {
			System.out.println("Strange number");
		} else if(myNumber == 2) {
			System.out.println("Unreachable code");
		} else {
			System.out.println("Will be printed");
		}
	}
    
}
\end{lstlisting}
\end{frame}

\begin{frame}{Vergleichsoperatoren}
Wie man Dinge vergleicht:
\begin{itemize}
    \item \texttt{==} Equal
    \item \texttt{!=} Not Equal
    \item \texttt{>} Greater Than (Größer als)
    \item \texttt{>=} Greater or Equal than (Größer gleich)
\end{itemize}
\textit{Note}: Man kann mehrere Vergleiche\\ mit \texttt{\&\&} (AND) or \texttt{||} (OR) verknüpfen.
\end{frame}

\subsection{for}
\begin{frame}[fragile]{for}
\begin{lstlisting}
for(initial value, condition, change) {
	// do code while condition is true
}
\end{lstlisting}
\end{frame}

\begin{frame}[fragile]{for example}
\begin{lstlisting}
public class ForExample {

	public static void main(String[] args) {
		for(int i = 0; i <= 10; i++) {
			System.out.print("na ");
		}
		System.out.println("BATMAN!");
	}
    
}
\end{lstlisting}
\end{frame}

\subsection{while}
\begin{frame}[fragile]{while}
\begin{lstlisting}
while(condition) {
	// do code while condition is true
}
\end{lstlisting}
\end{frame}

\begin{frame}[fragile]{while example}
\begin{lstlisting}
public class WhileExample {

	public static void main(String[] args) {
		int a = 0;
		while(a <= 10) {
			System.out.println(a);
            a++; // Otherwise you would get an endless loop
		}
	}
    
}
\end{lstlisting}
\end{frame}

\section{Methoden}
\begin{frame}{Was ist eine Methode?}
	Was ist eine Methode?
	\begin{itemize}
		\item<1-> ein Stück Code zum Wiederverwenden
		\item<2-> kann Daten von Außen bekommen
		\item<3-> kann Daten nach Außen zurückgeben
	\end{itemize}
	\onslide<4->{Andere Namen
	\begin{itemize}
		\item function
		\item procedure
		\item subroutine
	\end{itemize}
}

\end{frame}

\begin{frame}{Warum sollten man Methoden verwenden?}
	Warum sollten man Methoden verwenden?
	\begin{itemize}
		\item<1-> Programmierer sind faul $\rightarrow$ Mehr mit weniger erreichen
		\item<2-> bessere Struktur und weniger Änderungen
		\item<3-> reduziert Fehler
		\item<4-> wichtig für OOP
	\end{itemize}
	
\end{frame}

\begin{frame}[fragile, allowframebreaks] {Einführung in die \emph{Methoden}}
	Die einfachste Methode
	\begin{lstlisting}
	static void helloMethod(){
		System.out.println("Hello, method!");	
	}
	\end{lstlisting}
\framebreak
	Eine Methode aufrufen
	\begin{lstlisting}
	class Hello{
		public static main(String[] args){
			helloMethod();
		}
		static void helloMethod(){
			System.out.println("Hello, method!");	
		}
	}
	\end{lstlisting}

\framebreak
	Daten an eine Methode übergeben (Parameter)
	\begin{lstlisting}
		static void printHello(String input){
			System.out.println("Hello, " + input + "!");	
		}
	\end{lstlisting}

\framebreak
	Daten aus einer Methode zurückgeben (Return)
	\begin{lstlisting}
		static String getHello(String input){
			String hello = "Hello, " + input + "!";	
			return hello;
			// return "Hello, " + input + "!";
		}
	\end{lstlisting}

\framebreak
DEMO
	

\end{frame}

\begin{frame}{Übung}
	Gebe alle geraden Zahlen zischen 1 und 100/n aus. (n - User input)
	\begin{itemize}
		\item<2-> benutze while und if
		\item<3-> benutze for (und nicht if)
		\item<4-> benutze eine Methode zur Überprüfung
		
	\end{itemize}
\end{frame}


\begin{frame}
	\begin{center}
		\includegraphics[angle=-90, width=\textwidth]{0001}
	\end{center}	

\end{frame}



\end{document}