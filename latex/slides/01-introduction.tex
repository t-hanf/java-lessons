\input{../templates/course_definitions}
% This Document contains the information about this course.

% Authors of the slides
\author{Tobias Hanf, Manik Khurana}

% Name of the Course
\institute{Java-Course}

% Fancy Logo
\titlegraphic{\hfill\includegraphics[height=1.25cm]{../templates/fsr_logo_cropped}}


\title{Java}
\subtitle{Introduction}
\date{\today}


\lstdefinestyle{base}{
	%language=C,
	%emptylines=1,
	%breaklines=true,
	%basicstyle=\ttfamily\color{black},
	moredelim=**[is][\color{red}]{@}{@},
}


\begin{document}

% \section{Organisation}
\begin{frame}
	\titlepage
\end{frame}
\begin{frame}{Overview}
	\setbeamertemplate{section in toc}[sections numbered]
	\tableofcontents
\end{frame}

\section{Proceeding}
\begin{frame}{About this course}
% 	Language?\\
	Requirements
	\begin{itemize}
		\item You know how to use a computer
        \item Please bring your computer with You
% 		\item Java is not your first programming language or you are a fast learner
	\end{itemize}
	Proceeding
	\begin{itemize}
		\item There will be around 14 lessons
		\item Each covers a topic and comes with excercises
	\end{itemize}
\end{frame}

\begin{frame}{Question}
	What is your programming knowledge?
	\begin{itemize}
		\item None
		\item Tried it once or twice
		\item Basic knowledge
		\item Advanced
		\item 1337 hax0r
	\end{itemize}
\end{frame}

% \subsection{Resources}
\begin{frame}{Some resources}
	\begin{itemize}
		\item You can ask your tutor
		\item Book: Head first Java \\
			 \url{https://katalog.slub-dresden.de/id/0-1680506722}
		\item StackOverflow, FAQs, Online-tutorials, ... \hfill \\
		\item Official documentation \hfill \\
			\url{https://docs.oracle.com/javase/8/}
		\item Material-Repository \\
			\url{https://github.com/t-hanf/java-lessons}
	\end{itemize}
\end{frame}

\begin{frame}{Communication}
	\begin{itemize}
		\item Write an email to \\
			 \href{mailto:tobias.hanf@mailbox.tu-dresden.de}{tobias.hanf@mailbox.tu-dresden.de} \\
			 \href{mailto:tobias.hanf@mailbox.tu-dresden.de}{tobias.hanf@mailbox.tu-dresden.de}
	 	\item Use the "Contact Teacher" button
	 	\item Write an email to the programming mailing list (all tutors) \\
	 		\href{mailto:programmierung@ifsr.de}{programmierung@ifsr.de}
 		\item Other options, maybe Discord, Matrix, Messenger??
	 	
	\end{itemize}
\end{frame}

\begin{frame}{About Java}
	% \tikzoverlay at (7cm,1.4cm) {
	% 	\includegraphics[width=3cm, height=3cm]{res/logo-java.png}
	% }
	Pros:
	\begin{itemize}
		\item Syntax like C++
		\item Strongly encourages OOP
		\item Platform-independent (JVM)
		\item Very few external libraries
		\item[] $->$ Easy to use and very little to worry about
	\end{itemize}
\end{frame}

\begin{frame}{About Java}
	% \tikzoverlay at (7cm,1.4cm) {
	% 	\includegraphics[width=3cm, height=3cm]{res/logo-java.png}
	% }
	Cons:
	\begin{itemize}
		\item A lot of unnecessary features \\ in the JDK
		\item Slower than assembly
		\item No multi-inheritance
		\item Weak generics
		\item Mediocre support for other programming paradigms
		\item[] $->$ Neither fast, small nor geeky
	\end{itemize}
\end{frame}

\section{Your first program}
\begin{frame}{Hello World}
  DEMO
\end{frame}

\begin{frame}[fragile]{Creating your Working Environment}
  Open the Terminal
  \begin{lstlisting}
      mkdir myProgram
      cd myProgram
      touch Hello.java
      vim Hello.java
  \end{lstlisting}
\end{frame}

\subsection{Hello World!}

\begin{frame}[fragile]{Hello World!}
	This is an empty JavaClass.
  Java Classes always start with a capital letter
	\begin{lstlisting}
	public class Hello {

	}
	\end{lstlisting}
\end{frame}

\begin{frame}[fragile]{Hello World!}
	This is a small program printing \emph{Hello World!} to the console:
	\begin{lstlisting}
	public class Hello {
	    public static void main(String[] args) {
	        System.out.println("Hello World!");
	    }
	}
	\end{lstlisting}
\end{frame}

\begin{frame}[fragile]{How to run your program}
   save your program by pressing 'esc', then ':w'
   exit vim by typing ':q' (and hit return)
   then:
   \begin{lstlisting}[language=bash]
      javac Hello.java
      java Hello
   \end{lstlisting}
\end{frame}

\subsection{Setting up VSCode}

\begin{frame}{Hello World in an IDE}
   DEMO
\end{frame}

\begin{frame}{Receive a copy of IntelliJ IDEA}
  Visual Studio Code is a extensible code editor
  \begin{itemize}
      \item You can download VSCode at  \\
          \url{https://code.visualstudio.com/}
      \item Install the Java extension \\
       "Language Support for Java(TM) by Red Hat"
  \end{itemize}
  Other code editors or IDEs are also fine but you have to know how to use them.
\end{frame}

\section{Basics}

\begin{frame}{Programming}
	What is programming ?
	\begin{itemize}
		\item<1-> telling a computer what to do
		\item<2-> different instructions
		\item<3-> store a date (memory)
		\item<4-> do something with this date (compute)
		\item<5-> list instruction in order
	\end{itemize}
\end{frame}

\begin{frame}[fragile]{Code concepts}
	\begin{lstlisting}
		public class Hello {
			// Calculates some stuff and outputs everything on the console
			public static void main(String[] args) {
				System.out.println(9 * 23);
			}
		}
	\end{lstlisting}
\end{frame}

\begin{frame}[fragile]{Code concepts}
	\begin{lstlisting}
		public class Hello {
			// Calculates some stuff and outputs everything on the console
			public static void main(String[] args) {	        
				int x;
				x = 9;
				int y = 23;
				int z;
				z = x * y;
				
				System.out.println(z);
			}
		}
	\end{lstlisting}
\end{frame}

\begin{frame}[fragile]{Comments}
   \begin{lstlisting}
   public class Hello {
       // prints a "Hello World!" on your console
       public static void main(String[] args) {
           System.out.println("Hello World!");
       }
   }
   \end{lstlisting}
   You should always comment your code. \\
   Code is read more often than it is written.
   \begin{itemize}
       \item // single line comment
       \item /* comment spanning \\
           multiple lines */
   \end{itemize}
\end{frame}

\subsection{Some definitions}

\begin{frame}[fragile]{About the Semicolon}
	\begin{lstlisting}[style=base]
	public class Hello {
	    // prints a "Hello World!" on your console
	    public static void main(String[] args) {
	        System.out.println("Hello World!") @;@
	    }
	}
	\end{lstlisting}
	Semicolons conclude all statements. \\
	Blocks do not need a semicolon.
\end{frame}

\begin{frame}[fragile]{Blocks}
	\begin{lstlisting}[style=base]
	public class Hello @{@
	    // prints a "Hello World!" on your console
	    public static void main(String[] args) {
	        System.out.println("Hello World!");
	    }
	@}@
	\end{lstlisting}
	Everything between \{ and \} is a \emph{block}. \\
	Blocks may be nested.
\end{frame}

\begin{frame}[fragile]{Naming of Variables}
	\begin{itemize}
		\item The names of variables can begin with any letter or underscore. \\
		Usually the name starts with small letter.
		\item Compound names should use CamelCase.
		\item Use meaningful names.
	\end{itemize}
	\begin{lstlisting}
	public class Calc {
	    public static void main(String[] args) {
	    	int a = 0; // not very meaningful
	    	float myFloat = 5.3f; // also not meaningfull
	    	int count = 7; // quite a good name

	    	int rotationCount = 7; // there you go
	    }
	}
	\end{lstlisting}
\end{frame}

\begin{frame}{Primitive data types}
	Java supports some primitive data types:
	\begin{itemize}
		\item[boolean] a truth value (either \textbf{true} or \textbf{false})
		\item[int] a 32 bit integer
		\item[long] a 64 bit integer
		\item[float] a 32 bit floating point number
		\item[double] a 64 bit floating point number
		\item[char] an unicode character
		\item[void] the empty type (needed in later topics)
	\end{itemize}
\end{frame}

\subsection{Calculating}

\begin{frame}[fragile, allowframebreaks]{Calculating with \emph{int}}
	\begin{lstlisting}
	public class Calc {
	    public static void main(String[] args) {
	        int a;
	        a = 7;
	        System.out.println(a);
	        a = 8;
	        System.out.println(a);
	        a = a + 2;
	        System.out.println(a);
	    }
	}
	\end{lstlisting}
	
\framebreak
\begin{lstlisting}
	public class Calc {
		public static void main(String[] args) {
			int a; // declare variable a
			a = 7; // assign 7 to variable a
			System.out.println(a); // prints: 7
			a = 8;
			System.out.println(a); // prints: 8
			a = a + 2;
			System.out.println(a); // prints: 10
		}
	}
\end{lstlisting}
After the first assignment the variable is initialized.
\framebreak
\begin{lstlisting}
	public class Calc {
		public static void main(String[] args) {
			int a = -9;
			int b;
			b = a;
			System.out.println(a);
			System.out.println(b); 
			a++;
			System.out.println(a); 
		}
	}
\end{lstlisting}
\framebreak
	\begin{lstlisting}
	public class Calc {
	    public static void main(String[] args) {
	        int a = -9; // declaration and assignment of a
	        int b; // declaration of b
	        b = a; // assignment of b
	        System.out.println(a); // prints: -9
	        System.out.println(b); // prints: -9
	        a++; // increments a
	        System.out.println(a); // prints: -8
	    }
	}
	\end{lstlisting}
% \framebreak
% 	\begin{lstlisting}
% 	public class Calc {
% 	    public static void main(String[] args) {
% 	        int b; // declaration of b
% 	        System.out.println(b);
% 	    }
% 	}
% 	\end{lstlisting}
% 	Uninitialized variables will cause an Exception. \\
% 	An Exception is a kind of error we will discuss later.\\
% 	\vspace{1em}
% 	\emph{Always assign your variables!}
% \framebreak

\framebreak
	Some basic mathematical operations:
	\begin{tabular}{ll}
		Addition & \texttt{a + b;} \\
		Subtraction & \texttt{a - b;} \\
		Multiplication &\texttt{a * b;} \\
		Division & \texttt{a / b;} \\
		Modulo & \texttt{a \% b;} \\
		Increment & \texttt{a++;} \\
		Decrement & \texttt{a--;} \\
	\end{tabular}
\end{frame}

\begin{frame}[fragile, allowframebreaks]{Calculating with \emph{float}}
	\begin{lstlisting}
	public class Calc {
	    public static void main(String[] args) {
	        float a = 9;
	        float b = 7.5f;
	        System.out.println(a); // prints: 9.0
	        System.out.println(b); // prints: 7.5
	        System.out.println(a + b); // prints: 16.5
	    }
	}
	\end{lstlisting}

\framebreak
\begin{lstlisting}
	public class Calc {
		public static void main(String[] args) {
			float a = 0.1f;
			float b = 0.2f;

			System.out.println(((a + b) == 0.3));
		}
	}
\end{lstlisting}


\framebreak
	\begin{lstlisting}
	public class Calc {
		public static void main(String[] args) {
			float a = 0.1f;
			float b = 0.2f;
			
			System.out.println(((a + b) == 0.3)); // false
			System.out.println((a + b));
		}
	}
	\end{lstlisting}
	Float has a limited precision. \\
	\emph{This might lead to unexpected results!}
\end{frame}

\begin{frame}[fragile]{Mixing \emph{int} and \emph{float}}
	\begin{lstlisting}
	public class Calc {
	    public static void main(String[] args) {
	        float a = 9.3f;
	        int b = 3;
	        System.out.println(a + b); // prints: 12.3
	        float c = a + b;
	        System.out.println(c); // prints: 12.3
	    }
	}
	\end{lstlisting}
	Java converts from \textbf{int} to \textbf{float} by default, if necessary. \\
	But not vice versa.
\end{frame}

\subsection{Text with Strings}

\begin{frame}[fragile]{Strings}
	A String is not a primitive data type but an object. \\
	We discuss objects in detail in the next section.
	\begin{lstlisting}
	public class Calc {
	    public static void main(String[] args) {
	        String hello = "Hello World!";
	        System.out.println(hello); // print: Hello World!
	    }
	}
	\end{lstlisting}
\end{frame}

\begin{frame}[fragile]{Concatenation}
	\begin{lstlisting}
	public class Calc {
	    public static void main(String[] args) {
	        String hello = "Hello";
	        String world = " World!";
	        String sentence = hello + world;
	        System.out.println(sentence);
	        System.out.println(hello + " World!");
	    }
	}
	\end{lstlisting}
	You can concatenate Strings using the +. Both printed lines look the same.
\end{frame}

\begin{frame}[fragile]{Strings and Numbers}
	\begin{lstlisting}
	public class Calc {
	    public static void main(String[] args) {
	    	int factorA = 3;
	    	int factorB = 7;
	    	int product = factorA * factorB;
	    	String answer =
	            factorA + " * " + factorB + " = " + product;
	        System.out.println(answer); // prints: 3 * 7 = 21
	    }
	}
	\end{lstlisting}
	Upon concatenation, primitive types will be replaced by their current value as \emph{String}.
\end{frame}

\end{document}
